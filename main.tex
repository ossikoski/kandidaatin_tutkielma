%%%%%%%%%%%%%%%%%%%%%%%%%%%%%%%%%%%%%%%%%%%%%%%%%%%%%%%%%%%
%% Congratulations, you've made an excellent choice
%% of writing your Tampere University thesis using
%% the LaTeX system. This document attempts to be
%% as complete a template as possible to let you focus
%% in the most important part: the writing itself.
%% Thus the details regarding the visual appearance
%% and even structure have already been worked out
%% for you!
%%
%% I sincerely hope you will find this template useful
%% in completing your thesis project. I've tried to
%% add comments (followed by the % sign) to clarify
%% the structure and purpose of some of the commands.
%% Most of the magic happens in the file ./tauthesis,
%% which you are more than welcome to take a look at.
%% Just refrain from editing it in the most crucial
%% versions of the thesis!
%%
%% I wish you and your thesis project the best of luck!
%% If you've got any suggestions for improving this template,
%% please contact me via email at
%%
%% ville.koljonen (at) tuni.fi.
%%
%% Yours,
%%
%% Ville Koljonen
%% 16th May 2019
%%
%% PS. This template or its associated class file don't
%% come with a warranty. The content is provided as-is,
%% without even the implied promise of fitness to the
%% mentioned purpose. You, as the author of the thesis,
%% are responsible for the entire work, including the
%% provided material. No one else is liable to you for
%% any damage inflicted on you or your thesis were it
%% caused by using this template or not.
%%%%%%%%%%%%%%%%%%%%%%%%%%%%%%%%%%%%%%%%%%%%%%%%%%%%%%%%%%%

%%%%% PREAMBLE %%%%%

%%%%% Document class declaration.
% The possible optional arguments are
%finnish - thesis in Finnish (default)
%english - thesis in English
%numeric - citations in numeric style (default)
%authoryear - citations in author-year style
%draft - for faster non-final works, also skips images
%           (recommended, remove in the final version)
%programs - if you wish to display code snippets
% Example: \documentclass[english, authoryear]{tauthesis}
%          thesis in English with author-year citations
\documentclass[finnish, numeric, draft]{tauthesis}

% The glossaries package throws a warning:
% No language module detected for 'Finnish'.
% You can safely ignore this. All other
% warnings should be taken care of!

%%%%% Your packages.
% Before adding packages, see if they can be found
% in ./tauthesis already. If you're not sure that
% you need a certain package, don't include it in
% the document! This can dramatically reduce
% compilation time.

\usepackage{todonotes}
% Add [disable]{todonotes} to remove todonotes and remove the setlength

% Graphs
% \usepackage{pgfplots}
% \pgfplotsset{compat=1.15}

% Subfigures and wrapping text
% \usepackage{subcaption}

% Mathematics packages
\usepackage{amsmath, amssymb, amsthm}
%\usepackage{bm}

% The chemistry packages
% \usepackage{chemfig}
% \usepackage[version=4]{mhchem}

% Text hyperlinking
% \usepackage{hyperref}
% \hypersetup{hidelinks}

% (SI) unit handling
\usepackage{siunitx}

\sisetup{
    detect-all,
    math-sf=\mathrm,
    exponent-product=\cdot,
    output-decimal-marker={,} % for theses in FINNISH!
}

%%%%% Your commands.

% Print verbatim LaTeX commands
\newcommand{\verbcommand}[1]{\texttt{\textbackslash #1}}


% Basic theorems in Finnish and in English.
% Remove [chapter] if you wish a simply
% running enumeration.
% \newtheorem{lause}{Lause}[chapter]
% \newtheorem{theorem}[lause]{Theorem}

% \newtheorem{apulause}[lause]{Apulause}
% \newtheorem{lemma}[lause]{Lemma}

% Use these versions for individually
% enumerated lemmas
% \newtheorem{apulause}{Apulause}[chapter]
% \newtheorem{lemma}{Lemma}[chapter]

% Definition style
% \theoremstyle{definition}
% \newtheorem{maaritelma}{Määritelmä}[chapter]
% \newtheorem{definition}[maaritelma]{Definition}
% examples in this style

%%%%% Glossary information.

\loadglsentries[main]{tex/sanasto.tex}
\makeglossaries

%%%%% in citation information.

\addbibresource{tex/library.bib}

\begin{document}

%%%%% FRONT MATTER %%%%%

\frontmatter

%%%%% Thesis information and title page.

% The titles of the work. If there is no subtitle,
% leave the arguments empty. Pass the title in
% the primary language as the first argument
% and its translation to the secondary language
% as the second.
\title{Tarkkuuden parantaminen robotilla tarjoiltavien juomien kaatomäärissä}{Otsikko englanniksi}
\subtitle{Käytännön koe Drinkkirobotti-sovelluksella}{Alaotsikko englanniksi}

% The author name.
\author{Ossi Koski}

% The examiner information.
% If your work has multiple examiners, replace with
% \examiner[<label>]{<name> \\ <name>}
% where <label> is an appropriate (plural) label,
% e.g. Examiners or Tarkastajat, and <name>s are
% replaced by the examiner names, each on their
% separate line.
\examiner[Tarkastaja]{Tarkastaja}

% The finishing date of the thesis (YYYY-MM-DD).
\finishdate{2021}{01}{01}

% The type of the thesis (e.g. Kandidaatintyö
% or Master of Science Thesis) in the primary
% and the secondary languages of the thesis.
\thesistype{Kandidaatintyö}{Bachelor of Science Thesis}

% The faculty and degree programme names in
% the primary and the secondary languages
% of the thesis.
\facultyname{Tekniikan ja luonnontieteiden tiedekunta}{Faculty of Engineering and Natural Sciences}
\programmename{Teknisten tieteiden kandidaattiohjelma}{Bachelor's Programme in Engineering Sciences}

% The keywords to the thesis in the primary
% and the secondary languages of the thesis.
\keywords%
    {robotiikka}
    {robotics}

\maketitle

% Write the abstract(s) and the preface
% into a separate file for the sake of clarity.
% Pass the appropriate file name as the first
% argument to these commands. Put the \abstract
% in the primary language first and the
% \otherabstract in the secondary language second.
% Those who do not speak Finnish only need the
% first abstract. The second argument of
% the \preface command takes the place where
% the thesis was signed in.
\abstract{tex/tiivistelma.tex}
\otherabstract{tex/abstract.tex}
\preface{tex/alkusanat.tex}{Tampereella}

%%%%% Table of contents.

\tableofcontents

%%%%% Lists of figures, tables, listings and terms.

% Print the lists of figures and/or tables.
% Uncomment either of these commands as required.
% Both are optional, but if there are many important
% figures/tables, listing them may be a good idea.

% \listoffigures
% \listoftables
% \lstlistoflistings

% Print the glossary of terms.

\glossary

%%%%% MAIN MATTER %%%%%

\mainmatter

% Write each of the chapters of the thesis
% into a separate file for the sake of clarity.
% They can be \input as shown below. Give both
% the chapters and their files as descriptive
% names as possible.

\chapter{Johdanto}
\label{ch:johdanto}
Robotiikan käyttö maailmanlaajuisesti yleistyy koko ajan huomattavalla nopeudella. Roboteista tulee yhä älykkäämpiä ja taloudellisesti kannattavampia. Reilusti suurin osa roboteista on teollisuusrobotteja, jotka ovat kätketty tuotantolaitoksiin \cite{Heer2020}. Palvelurobotiikka kuitenkin kasvattaa myös osuuttaan, ja esimerkiksi robottiruohonleikkurit ja -imurit ovat aikaisempaan verrattuna suhteellisen tuttuja näkyjä kotitalouksissa. Myös esimerkiksi sote-aloilla palvelurobotteja käytetään jo melko laajalti \cite{Jyvaskylanyliopisto2018}. Yhä innovatiivisempia tapoja käyttää robotteja palvelualalla tulee esiin. Yksiä näistä ovat ravintola-alan sovellukset.

Juomia tarjoilevat robotit ovat toistaiseksi vielä melko uniikkeja. Juomien tarjoilu robotilla saattaa joissain tilanteissa nopeuttaa tarjoilua, mutta suurin myyntivaltti siinä on sen näyttävyys. Juomien automaattisessa tarjoilussa ongelmia aiheuttavat nesteen ominaisuudet. Usein robotit käyttävät hyväksi jonkinlaista juoma-automaattia \cite{Kelly2020} tai sitten pullot roikkuvat väärin päin esimerkiksi robotin päällä ja niissä on erityinen korkki, jota painamalla ja näin korkin venttiilin avaamalla robotti voi laskea pullosta juomaa \cite{Ro2016}. Näissä ratkaisuissa hyvä puoli on se, että nesteen virtaus tarjoiluastiaan on lähellä vakiota ja täten kokonaisjuomamäärä on helposti säädeltävissä. Tässä kandidaatintyössä käsitellään tapausta, jossa robotti kaataa juomaa pullosta mukiin. Tällöin juomankaatotehtävän voi antaa käytännössä mille tahansa robotille tarvitsematta juoma-automaatteja tai kokonaista robottisolua, jossa pullot roikkuvat robotin päällä. Pullosta kaataessa kuitenkin nesteen virtauksen ja pullon kallistuksen mukana tulevat muuttujat tekevät juomamäärän säätelystä hankalampaa.

Tässä kandidaatintyössä kerrotaan Pullonkaula ry:n Drinkkirobotti-sovelluksesta, ja siitä miten sillä on aikaisemmin toteutettu juomien kaato avonaisesta pullosta. Tämä tapa on sisältänyt ongelmia juuri kaatomäärän tarkkuuden suhteen, ja näitä ongelmia käsitellään alaluvussa \ref{ch:vanhan_ongelmat}. Tämän jälkeen kartoitetaan lyhyesti, millaisia ratkaisuja ongelmaan on kehitetty viime vuosina tehdyissä tutkimuksissa. Sen jälkeen valitaan ja perustellaan tässä työssä käytetty ratkaisu kaadon määrän ohjaamiseen. Lopuksi vertaillaan vanhan ja uuden kaatotavan käytössä saatuja tuloksia keskenään. Tämän työn pohjalta on mahdollista soveltaa kehitettyjä tekniikoita muihin samankaltaisiin sovelluksiin.


\chapter{Yhteenveto}
\label{ch:yhteenveto}
Työn tavoitteena oli kehittää Pullonkaula ry:n Drinkkirobotti\hyp{}sovellukselle tarkempi tapa kaataa juomia siten, että tuloksena saadaan oikea tilavuus juomaa. Työssä annettiin ensin yleiskuva Drinkkirobotti\hyp{}sovelluksesta ja sen rakenteesta ja ohjelmistoarkkitehtuurista. Sitten käsiteltiin aikaisemmin käytössä ollutta tapaa juomien määrän mittaamiseen ja kerrottiin, minkälaisia ongelmia sen käytössä on ollut. Juomien kaadossa ongelmia aiheuttaa mm. epälineaarisuus virtaavan juoman määrässä sen suhteen, miten paljon pullossa on juomaa jäljellä. Pahimmassa tapauksessa juomanokan korvausilmaputken tukkeutuminen aiheuttaa pulloon jäävän alipaineen takia sen, että asiakkaan lasi jäi lähes tyhjäksi. Tälle tapaukselle ei voitu vanhalla kaatotavalla muuta kuin tilata uusi juoma. Työssä perusteltiin, miksi kaadon tarkkuus on tärkeää, ja näihin ongelmiin pyrittiin löytämään ratkaisu työssä kehitetyllä uudella kaatotavalla.

Välissä kartoitettiin viime vuosina tehtyjä tutkimuksia aiheesta. Useampia eri ratkaisuja juomien kaadon tarkentamiseksi löydettiin. Näissä ratkaisuissa on käytetty robotin nivelen voima-anturia, haptista etäohjainta, kamerakuvaa tai jopa äänidataa kaadoista, sekä näiden yhdistelmiä. Näiden yhteydessä käytettiin monessa ratkaisussa koneoppimista esimerkiksi kamera- tai äänidatan prosessoinnissa ja niiden yhdistämisssä kaadettuun juomamäärään.

Tässä työssä päädyttiin ratkaisuun, jossa kaadon kohteena olevan juomalasin alle laitetaan vaaka. Tämän ratkaisun valintaan vaikutti sen helppous, sillä robottisolussa oli jo käytössä vaaka, jota pystyttiin hyödyntämään työssä. Lisäksi tällaista tapaa pystyisi helposti käyttämään tavallisissa robottisolussa, kunhan vain vaakaan saadaan yhteys esimerkiksi robottia ohjaavan logiikan kautta. Se ei siis aseta suuria vaatimuksia robottisolulle, eikä vaadi kallista laitteistoa. Uusi kaatotapa toimii siten, että robotin logiikka pyytää vaa'alta jatkuvasti kaadon aikana tuloksia painosta, ja käskee robottia lopettamaan kaadon, kun juomaa on tarpeeksi.

Uudella kaatotavalla päästiin hyviin tuloksiin. Kaikki testatut juomamäärät vaihtelivat korkeintaan yhdellä grammalla. Lisäksi saatiin ratkaistua kaatonokan korvausilmaputken tukkiutumiseen liittyvä ongelma, joka johti virtauksen estymisen takia siihen, että asiakkaan lasi jäi lähes tyhjäksi. Tämä ongelma ratkaistiin siten, että logiikka havaitsee jos juomaa ei virtaa lasiin ja käskee robottia tekemään uuden kaatoliikkeen. Uuden kaatotavan heikkous on se, että kaadon jälkeen pullon suoristuksen aikana virtaavaan juomamäärään ei pystytä vaikuttamaan. Pienten, alle kolmen senttilitran kokoisia kaatoja ei ole myöskään mahdollista tehdä. Näiden heikkouksien ratkaisemiseksi olisi mahdollista tehdä kaatokulman säätö, ja se voisi olla tulevaisuuden kehityskohde Drinkkirobotti\hyp{}sovellukseen.


%%%%% Bibliography/references.

% Print the bibliography according to the
% information in ./tex/references.bib
% and the in-line citations used in the body
% of the thesis.
% \emergencystretch=2em
\printbibliography[heading=bibintoc]

%%%%% Appendices.

% Use only if it clarifies the structure of
% the document. Remember to introduce each
% appendix.

%\begin{appendices}
%\end{appendices}

\end{document}
