Tässä alaluvussa käsitellään useamman vuoden käytössä ollutta ratkaisua juomien kaatamiseen Drinkkirobotilla. Ratkaisussa juomien kaadolle robottikoodissa on yksittäinen tehtäväohjelma, joka on esitetty alla.

\lstset{style=Yaskawatyyli}
\lstinputlisting[label={prog:POURDRINKS}, caption={POURDRINKS-tehtäväohjelma}]{code/POURDRINKS.txt}

Kaadon tarkkuutta tarkasteltaessa merkittävä osa tehtäväohjelmassa on ensimmäinen if-else-haara, joka alkaa ohjelman \ref{prog:POURDRINKS} riviltä 5. Se käsittelee kaadossa rivillä 24 käytetylle ajastimelle annetun arvon laskentaa. Haluttu kaadon määrä senttilitroina on asetettu muuttujaan I003. POURDRINKS-tehtäväohjelma laskee alussa ajan, jonka mukaan robotti kaataa juomaa. Kuten ohjelmasta \ref{prog:POURDRINKS} nähdään, kaadon aika sekunteina saadaan kaavalla
\begin{align}
   t = 60 \cdot V - 171 \mathrm{,}
\end{align}
jossa V on muuttuja I003 eli haluttu tilavuus senttilitroina. Kyseessä on lineaarinen yhtälö, jossa kulmakerroin 60 ja vakiotermi -171 on määritetty kokeellisesti kaatamalla eri määriä juomia. Ensimmäisen if-elsen tehtävä on myös tarkastaa, onko haluttu kaatomäärä alle kolme senttilitraa. Tätä pienemmillä määrillä yllä kuvattu lineaarinen funktio antaisi negatiivisen ajan. Niinpä kaikilla määrillä, jotka ovat alle kolme senttilitraa, kaatoajaksi on määritelty yksi sekunti.

Tehtäväohjelma toimii siis pääpiirteittäin siten, että robotti liikkuu kohteena olevan mukin kohdalle ja kallistaa pulloa. Kun pullo on kallistunut kaatoasentoon, käynnistyy ajastin, jonka kesto on määritelty muuttujaan I011.

Tämä tehtäväohjelma hoitaa kaadon useammalle mukille samalla, sillä siinä käytetään silmukkaa LABEL10- ja LABEL999-lippujen sekä muuttujien I004 ja I005 avulla. Muuttuja I005 on lasin numero, johon robotti sen hetkisellä iteraatiolla kaataa. Muuttuja I004 taas on haluttujen lasillisten määrä. Tehtäväohjelma CALCULATEPOUR laskee koordinaatit, joissa kohdemuki on ja tallettaa sen P-alkuisiin paikkamuuttujiin. Tässä se käyttää apuna muuttujaa I005. Mukipaikat ovat vaakasuorassa rivissä, joten siirryttäessä mukilta toiseen riittää vain lisätä robotin koordinaatiston x-koordinaattia mukien etäisyyden verran.
