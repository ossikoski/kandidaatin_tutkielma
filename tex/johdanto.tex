Robotiikan käyttö maailmanlaajuisesti yleistyy koko ajan huomattavalla nopeudella. Roboteista tulee yhä älykkäämpiä ja taloudellisesti kannattavampia. Reilusti suurin osa roboteista on teollisuusrobotteja, jotka ovat kätketty tuotantolaitoksiin \cite{Heer2020}. Palvelurobotiikka kuitenkin kasvattaa myös osuuttaan, ja esimerkiksi robottiruohonleikkurit ja -imurit ovat aikaisempaan verrattuna suhteellisen tuttuja näkyjä kotitalouksissa. Myös esimerkiksi sote-aloilla palvelurobotteja käytetään jo melko laajalti \cite{Jyvaskylanyliopisto2018}. Yhä innovatiivisempia tapoja käyttää robotteja palvelualalla tulee esiin. Yksiä näistä ovat ravintola-alan sovellukset.

Juomia tarjoilevat robotit ovat toistaiseksi vielä melko uniikkeja. Juomien tarjoilu robotilla saattaa joissain tilanteissa nopeuttaa tarjoilua, mutta suurin myyntivaltti siinä on sen näyttävyys. Juomien automaattisessa tarjoilussa ongelmia aiheuttavat nesteen ominaisuudet. Usein robotit käyttävät hyväksi jonkinlaista juoma-automaattia \cite{Kelly2020} tai sitten pullot roikkuvat väärin päin esimerkiksi robotin päällä ja niissä on erityinen korkki, jota painamalla ja näin korkin venttiilin avaamalla robotti voi laskea pullosta juomaa \cite{Ro2016}. Näissä ratkaisuissa hyvä puoli on se, että nesteen virtaus tarjoiluastiaan on lähellä vakiota ja täten kokonaisjuomamäärä on helposti säädeltävissä. Tässä kandidaatintyössä käsitellään tapausta, jossa robotti kaataa juomaa pullosta mukiin. Tällöin juomankaatotehtävän voi antaa käytännössä mille tahansa robotille tarvitsematta juoma-automaatteja tai kokonaista robottisolua, jossa pullot roikkuvat robotin päällä. Pullosta kaataessa kuitenkin nesteen virtauksen ja pullon kallistuksen mukana tulevat muuttujat tekevät juomamäärän säätelystä hankalampaa.

Tässä kandidaatintyössä kerrotaan Pullonkaula ry:n Drinkkirobotti-sovelluksesta, ja siitä miten sillä on aikaisemmin toteutettu juomien kaato avonaisesta pullosta. Tämä tapa on sisältänyt ongelmia juuri kaatomäärän tarkkuuden suhteen, ja näitä ongelmia käsitellään kappaleessa \ref{ch:vanhan_ongelmat}. Sen jälkeen käydään läpi sitä, miten on pyritty kehittämään parempi tapa ohjata kaadon määrää sovelluksessa. Lopuksi vertaillaan kahta tapaa keskenään. Vertailu on pyritty tekemään sekä juuri tämän robottisolun ominaisuuksiin liittyen että myös yleisemmällä tasolla. Näin tämän työn pohjalta on mahdollista soveltaa tuloksia muihin samankaltaisiin sovelluksiin.
