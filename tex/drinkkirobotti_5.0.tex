Drinkkirobotti 5.0 on Pullonkaula ry:n, eli Tampereen yliopistossa toimivan tuotantotekniikan ammattiainekerhon opiskelijoiden vapaa-ajalla tekemä projekti. Robottisolu koostuu Yaskawan HC-10-monitoimirobotin Teach Me-solusta ja sen ympärille rakennetuista baaritiskeistä. Baaritiskeissä on sisäpuolella pullohyllyt, ja kun robotilta tilataan juomaa, se hakee oikeat pullot hyllystä ja valmistaa niistä monikomponenttijuomia kaatamalla juomaa lasiin. Baaritiskillä on neljä paikkaa lasille. Robotilta juomien tilaaminen tapahtuu tabletille tehdyn käyttöliittymän avulla. Yaskawan HC-10 on yhteistyörobotti, eli se sisältää voima-antureita ja näiden avulla ihmisen ja robotin välinen yhteistyö on helpottunut. \cite{Pullonkaula2020}

Drinkkirobotin ohjelmisto koostuu useasta tasosta. Front end sisältää React.js:llä tehdyt käyttöliittymät operaattoria ja asiakasta varten. Operaattorinäkymästä voi esimerkiksi hallita robottisolun pullohyllyssä olevia pulloja lisäämällä tai poistamalla niitä ja valita robotin idle-tilassa tekemiä liikkeitä. Asiakasnäkymästä voi selata tarjolla olemia drinkkejä ja niiden sisältämiä komponentteja ja tilata drinkkejä kerrallaan 1-4 kappaletta. Back end toimii Node.js:llä ja se sisältää tiedon tarjolla olevista juomista ja niiden komponenteista sekä määristä. Siellä on myös reaaliaikainen tieto pullohyllyn tilanteesta, eli mitä pulloja hyllyssä on ja kuinka paljon juomaa missäkin pullossa on jäljellä. Kun operaattori lisää pullon, hän merkitsee käyttöliittymään pullossa olevan juoman määrän, ja tämä välitetään back endille. Kun pullosta kaadetaan juomaa, back endiin päivittyy uusi tieto pullossa olevasta juomasta. Jos pullossa on liian vähän juomaa jäljellä minkään drinkin tekoon, robotti poistaa pullon hyllystä automaattisesti.

Itse robottikoodi on suhteellisen yksinkertaista. Siinä käytetään Yaskawan Inform-ohjelmointikieltä, ja robottia ohjaa Yaskawan YRC-1000 -ohjausyksikkö. Robottikoodi sisältää "jobeja", jotka sisältävät robotin liikeratoja ja niiden parametrien määrittelyjä ja joissakin väleissä lyhyitä odotusaikoja. Robottisolun aivoina voidaan pitää sen bisneslogiikkatasoa. C\#-kielellä kirjoitettua logiikkaa pyörittää robottisolun sisällä oleva Rasperry Pi -tietokone. Logiikka ohjaa robottia kertomalla, mikä funktio, eli "job" milloinkin suoritetaan. Se siis käsittelee front endiltä saadut tilaukset ja päättää esimerkiksi mikä pullo haetaan ja mihin lasiin juomaa kaadetaan. Rasperry Pi ohjaa logiikan lisäksi robotin tarttujaa. Tarttujaa voidaan ohjata robottikoodista siihen tarkoitetuilla jobeilla, jolloin robottikoodi kutsuu Rasperry Pi:tä.

Uusimpana lisäyksenä robottisoluun on implementoitu vaaka, joka on niin sanotulla pullonvaihtopisteellä. Vaa'an toiminta on hyvä esimerkki robotti-bisneslogiikkarajapinnasta, sillä vaa'an voi kalibroida ja sen kullakin hetkellä näyttämää arvoa voi kysyä logiikan avulla. Tämän vaa'an avulla robotti tunnistaa, onko pullonvaihtopisteellä pulloa. Tätä käytetään esimerkiksi uuden pullon lisäämisessä: Kun robotti hakee uutta pulloa hyllyyn, niin se pysähtyy ensin pullonvaihtopisteen eteen. Jos vaaka kertoo, että pullonvaihtopiste on tyhjä, logiikka ei käske robottia hakemaan olematonta pulloa, vaan käskee sen odottamaan tietyn ajan. Robotti hakee pullon vasta, kun se lisätään pullonvaihtopisteelle. Tämän toiminnallisuuden avulla robotti ei siis vie olematonta pulloa hyllyyn, jolloin back endissä olisi tieto pullosta tietyllä kohdalla pullohyllyä, vaikka pulloa ei oikeasti ole. Samoin, jos robotti on poistamassa pulloa pullohyllystä, ei se vie sitä pullonvaihtopisteelle jos siellä on jo pullo. Näin robotti ei työnnä vanhaa pulloa vaihtopisteeltä alas, vaan odottaa että se poistetaan sieltä. Työn alla on myös ominaisuus, jossa logiikka kertoo back endille, miten paljon juomaa lisätyssä pullossa on, ja operaattorin tehtäväksi jää vain tarkistaa lukema, eikä itse arvioida lisättävässä pullossa olevaa juoman määrää.

Vaaka koostuu HX711 mittakortista ja siihen kytketystä 5kg painoanturista, ja näitä ohjaa arduino-kirjastoja apuna käyttäen ohjelmoitu NodeMCU V3. Kommunikaatio logiikkaan toimii WiFi-yhteydellä lähetettävillä UDP-paketeilla. Vaakaa pystyy siis siirtämään, ja niinpä tätä samaa vaakaa käytetään juomankaatosysteemin takaisinkytkentään, josta kerrotaan kappaleessa 3.
