Työn tavoitteena oli kehittää Pullonkaula ry:n Drinkkirobotti\hyp{}sovellukselle tarkempi tapa kaataa juomia siten, että tuloksena saadaan oikea tilavuus juomaa. Työssä annettiin ensin yleiskuva Drinkkirobotti\hyp{}sovelluksesta ja sen rakenteesta ja ohjelmistoarkkitehtuurista. Sitten käsiteltiin aikaisemmin käytössä ollutta tapaa juomien määrän mittaamiseen ja kerrottiin, minkälaisia ongelmia sen käytössä on ollut. Juomien kaadossa ongelmia aiheuttaa mm. epälineaarisuus virtaavan juoman määrässä sen suhteen, miten paljon pullossa on juomaa jäljellä. Pahimmassa tapauksessa juomanokan korvausilmaputken tukkeutuminen aiheuttaa pulloon jäävän alipaineen takia sen, että asiakkaan lasi jäi lähes tyhjäksi. Tälle tapaukselle ei voitu vanhalla kaatotavalla tehdä muuta kuin tilata uusi juoma. Työssä perusteltiin, miksi kaadon tarkkuus on tärkeää, ja kuvattuihin ongelmiin pyrittiin löytämään ratkaisu työssä kehitetyllä uudella kaatotavalla.

Välissä kartoitettiin viime vuosina tehtyjä tutkimuksia aiheesta. Useampia eri ratkaisuja juomien kaadon tarkentamiseksi löydettiin. Näissä ratkaisuissa on käytetty robotin nivelen voima-anturia, haptista etäohjainta, kamerakuvaa tai jopa äänidataa kaadoista, sekä näiden yhdistelmiä. Näiden yhteydessä käytettiin monessa ratkaisussa koneoppimista esimerkiksi kamera- tai äänidatan prosessoinnissa ja niiden yhdistämisessä kaadettuun juomamäärään.

Tässä työssä päädyttiin ratkaisuun, jossa kaadon kohteena olevan juomalasin alle laitetaan vaaka. Tämän ratkaisun valintaan vaikutti sen helppous, sillä robottisolussa oli jo käytössä vaaka, jota pystyttiin hyödyntämään työssä. Lisäksi tällaista tapaa pystyisi helposti käyttämään tavallisissa robottisolussa, kunhan vain vaakaan saadaan yhteys esimerkiksi robottia ohjaavan logiikan kautta. Se ei siis aseta suuria vaatimuksia robottisolulle, eikä vaadi kallista laitteistoa. Uusi kaatotapa toimii siten, että robotin logiikka pyytää vaa'alta jatkuvasti kaadon aikana tuloksia painosta, ja käskee robottia lopettamaan kaadon, kun juomaa on tarpeeksi.

Uudella kaatotavalla päästiin hyviin tuloksiin. Kaikki testatut juomamäärät vaihtelivat korkeintaan yhdellä grammalla. Lisäksi saatiin ratkaistua kaatonokan korvausilmaputken tukkiutumiseen liittyvä ongelma, joka johti virtauksen estymisen takia siihen, että asiakkaan lasi jäi lähes tyhjäksi. Tämä ongelma ratkaistiin siten, että logiikka havaitsee jos juomaa ei virtaa lasiin ja käskee robottia tekemään uuden kaatoliikkeen. Uuden kaatotavan heikkous on se, että kaadon jälkeen pullon suoristuksen aikana virtaavaan juomamäärään ei pystytä vaikuttamaan. Pienten, alle kolmen senttilitran kokoisia kaatoja ei ole myöskään mahdollista tehdä. Näiden heikkouksien ratkaisemiseksi olisi mahdollista tehdä kaatokulman säätö, ja se voisi olla tulevaisuuden kehityskohde Drinkkirobotti\hyp{}sovellukseen.
