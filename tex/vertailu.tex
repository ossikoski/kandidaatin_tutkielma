Työssä tutkittiin sekä vanhan että uuden kaatoratkaisun pohjalta saatuja tuloksia kaatojen määrissä, jotta niitä voitaisiin vertailla. Kaatoja tehtiin 4cl, 10cl, 16cl ja 20cl kokoisilla tilauksilla. Samalla jokaisen kaadon välissä punnittiin pullossa jäljellä oleva neste, jotta voidaan huomioida sen vaikutus kaadon määrään. Ideaalisesti pullossa olevan nesteen määrä ei vaikuttaisi kaadon määrään.

Alla on esitetty kuvaajassa vanhan kaatoratkaisun tulokset eri kokoisilla tilauksilla.

\todo{kuvaaja}

Kuten kuvaajasta huomataan, varianssia on kohtalaisen paljon. Saadut juomamäärät laseissa eivät ole tasaisia, vaan joka kaadolla muuttuu hieman. Esimerkiksi 4cl tilauksessa suurin saatu juomamäärä oli noin 50 grammaa ja pienin n. 36 grammaa. 20cl tilauksessa suurin saatu juomamäärä oli noin 255 grammaa ja pienin n. 214 grammaa. Näiden välinen ero on huomattava kun tarkastellaan sen prosentuaalista osuutta halutusta juomamäärästä.

Kuvaajassa olevista tuloksista saadaan myös vahvistus väitteelle, että pullossa jäljellä oleva juomamäärä vaikuttaa saatuun juoman määrään. Usealla kaatomäärällä mukiin saatu määrä vaihtelee ensin satunnaisesti tietyllä välillä, mutta käyrän loppupää kääntyy alaspäin. Tästä voidaan päätellä, että

Sen lisäksi, että saadut juomamäärät vaihtelevat keskenään paljon ja pullossa oleva juomamäärä vaikuttaa saadun juoman määrään, kaatomäärät ovat myös keskimäärin liian suuria. Koska jokaisella kaatomäärällä juomaa on tullut liikaa suhteessa tilattuun määrään, voidaan päätellä että vanhan funktion kulmakerroin on ollut liian suuri.

Seuraavaksi alla on esitetty kuvaajassa uuden kaatoratkaisun tulokset samankokoisilla tilauksilla.
