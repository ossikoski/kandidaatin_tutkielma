Edellisessä kappaleessa kuvattu ratkaisu on toiminut kohtalaisen hyvin ottaen huomioon sen yksinkertaisuuden. Siinä on kuitenkin selkeitä heikkouksia. On selvää esimerkiksi, että alle kolmen senttilitran kaatoja tällä toteutuksella ei käytännössä voi tehdä. Tämä toteutus ei myöskään ota huomioon mitään muita muuttujia mitä juomien kaadossa voi olla halutun juomamäärän lisäksi. Tässä kappaleessa käydään läpi erilaisia ongelmia, joita vanhassa kaatoratkaisussa on ollut. \todo{Metateksti taas?}

Eri muuttujien vaikutusta juoman kaatoon on yritetty vähentää käyttämällä aina samanlaisia pulloja, ja sen lisäksi pulloissa kaatonokkia. Tämä helpottaa myös kaatoliikkeen ohjelmointia robotille, kun pullosta tuleva juoman virtaus osuu mukiin helpommin. Virtaus on myös tasaisempi. Kaatonokissakin on kuitenkin ongelmia. Kaatonokassa on pieni korvausilmaputki, joka tasaa painetta pullon sisä- ja ulkopuolella, ja näin virtaukseen ei tule nykivää liikettä. Kaatoja pullosta ilman kaatonokkaa ovat simuloineet Geiger et al. \cite{Geiger2012}. Simulaatiosta huomataan, miten ilman kaatonokkaa ilman pitää virrata kaatoaukosta välillä sisäänpäin. Juoman kaatamisessa kaatonokan kanssa robotilla on kuitenkin myös omat ongelmansa. Suurin niistä on se, että korvausilmaputkeen saattaa mennä pisara nestettä. Näin pullon sisään ei tule korvausilmaa, ja kaatonokasta ei virtaa nestettä liian suuren paine-eron takia, vaikka pullo olisi kokonaan ylösalaisin. Tällaista tilannetta ei robotti ole osannut havaita. Sen takia asiakkaan lasi on jäänyt käytännössä tyhjäksi ja juomatilaus on jouduttu tekemään uudelleen. Tavoite uudelle kaatoratkaisulle on tällaisen tilanteen tunnistaminen ja siitä palautuminen. \todo{kuva kaatonokasta?}

Juomien kaatoon muuttujia tuo reilusti myös se, millaista juomaa kaadetaan. Veden, mehujen ja muiden kuohumattomien ja viskositeetiltään samankaltaisten juomien kaataminen onnistuu vielä melko hyvin lineaarisen funktion mukaan, mutta ongelmia alkaa tulla varsinkin kuohuvien juomien kanssa. Tällöin saman lineaarisen funktion käyttö ei enää toimi, sillä esimerkiksi kuplat aiheuttavat turbulenssia kaadon ajan.

Ongelmat kaatojen määrässä kertautuvat, kun robotin backend saa väärää tietoa pulloista kaadetuista määristä. Jos robottia kutsutaan kaatamaan esimerkiksi kymmenen senttilitraa juomaa, ja juomaa kaatuukin senttilitra liikaa, niin backend laskee pullosta lähteneen silti kymmenen senttilitraa. Kun tämä tapahtuu tarpeeksi monta kertaa, niin backendin tieto pulloissa jäljellä olevista juomista ei pidä enää paikkaansa. Pahimmillaan backendissä voi olla tieto, että pullosta riittäisi juomaa vielä yhteen lasilliseen, mutta todellisuudessa kyseinen pullo on lähes tyhjä. Tällöin jää operaattorin vastuulle huomata, että juoma pullossa ei riitä. Näin voidaan taas joutua tilanteeseen, jossa asiakas tilaa juomaa, mutta saakin vain tyhjän lasin.

Robotin logiikassa on ominaisuus, joka tarkistaa, onko pullossa liian vähän juomaa jäljellä ja se kutsuu suoraan pullon poistoa solusta. Jotta ei päädyttäisi tilanteeseen, missä lähes tyhjästä pullosta yritetään tarjoilla juomaa, on jouduttu määrittelemään palautettavan pullon jäljellä olevaksi juomamäärän rajaksi melko korkea määrä. Tällöin robotti poistaa solusta pulloja, jossa juomaa olisi vielä tarpeeksi, koska ei haluta että joudutaan tilanteeseen, joka haittaa asiakasta. Uudella kaatoratkaisulla tavoitellaan tilannetta, jossa backendin tieto pulloissa jäljellä olevista juomamääristä on koko ajan melko tarkka. Tällöin robotti voisi myös toteuttaa kaadot siten, että se kaataa vanhan pullon tyhjäksi, hakee uuden pullon ja jatkaa kaatoa siitä.
