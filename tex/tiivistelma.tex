Juomien kaataminen robotilla tarkasti on nesteen ominaisuuksista ja epälineaarisuuksista johtuen haastavaa, vaikka sekä kaato- että kohdeastiat olisivat aina samoja. Tarjoiltavan juomamäärän kuitenkin tulisi olla tarkka lakisääteisistä syistä ja asiakkaan tyytyväisyyden takia. Tässä kandidaatintyössä on pyritty toteuttamaan tarkka kaatosysteemi juomien kaadolle robotilla.

Työssä käsitellään Pullonkaula ry:n Drinkkirobotti 5.0\hyp{}sovellusta, ja sovelluksessa aikaisemmin käytettyä kaatoratkaisua, joka perustui oletukseen lineaarisuudesta kaatoajan ja kaadetun juomamäärän välillä. Samalla käsitellään vanhan ratkaisun sisältämiä ongelmia. Yksi havaittu ongelma oli kaadetun määrän tarkkuuden lisäksi juomanokan korvausilmaputken tukkeutuminen, joka johti siihen, ettei alipaineen takia pullosta virrannut juomaa. Vanhan kaatoratkaisun toimintaa on havainnollistettu mittaamalla kaadettuja määriä ja esittämällä ne kuvaajassa.

Työ sisältää lyhyen katsauksen mahdollisiin tapoihin, joilla ongelmaan on etsitty ratkaisua viime vuosina tehdyissä tutkimuksissa. Tähän työhön valitussa ratkaisussa mitataan vaa'an avulla jo kaadettua juomamäärää ja käytetään sitä takaisinkytkentänä robotin logiikalle. Logiikka pyytää vaa'alta jatkuvasti tuloksia mukiin kaadetun juoman painosta. Logiikka pystyy sitten käskemään robottia lopettamaan kaadon sopivalla hetkellä. Tämän tavan etuna oli sen yksinkertaisuus: Se ei aseta suuria vaatimuksia robottisolulle eikä vaadi esimerkiksi kalliita antureita.

Työssä kehitetyllä ratkaisulla päästiin tavoitteeseen, eli yksittäisten kaatojen tuloksena olevien juomamääärien hajonta pieneni merkittävästi. Juomien kaatoa testattiin vedellä ja kaikissa tapauksissa juomamäärät olivat maksimissaan yhden gramman eli yhden millilitran päässä tavoitteesta. Myös nämä tulokset on esitetty kuvaajassa, ja niitä voi vertailla vanhalla tavalla tehtyjen kaatojen tuloksiin. Työssä saatiin kehitettyä myös ratkaisu tapaukselle, jossa kaatonokan korvausilmaputken tukkiutumisesta johtuen pullosta ei virrannut juomaa. Tämä ongelma pystytään tunnistamaan vaa'an avulla ja sitten voidaan yrittää uutta kaatoa. Sekä kaatomäärien tarkkuutta että kaatojärjestelmän vikasietoisuutta virhetilanteiden edessä saatiin siis parannettua verrattuna vanhaan ratkaisuun.
