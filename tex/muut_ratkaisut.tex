Tässä kappaleessa kartoitetaan mahdollisia ratkaisuja kappaleessa \ref{ch:vanhan_ongelmat} esitetyille ongelmille, valitaan työssä käytetty ratkaisu ja perustellaan sen valinta.

Kuten kappaleessa \ref{ch:vanhan_ongelmat} todettiin, myös Suomen alkoholilaki määrittää alkoholin perusannokset. Lakisääteisyys alkoholin tarjoilussa on saanut aikaan erilaisia ratkaisuja mitata tarjoiltavaa juomaa. Jopa englanninkielinen Wikipedia-artikkeli määrittelee alkoholijuomien mittaamistavat omaan luokkiinsa \cite{Wikipedia}. Mitta-astioiden ja tavallisten, kuvassa \ref{fig:kaatonokka} esitettyjen kaatonokkien lisäksi on olemassa juoman virtausta mittaavia kaatonokkia. Kuten johdantokappaleessa todettiin, on korkkeja, joissa on pieni tila johon juomaa virtaa tietty määrä, ja sitten korkin venttiilin avaamalla tuon määrän voi laskea tilasta ulos. Tämän jälkeen tila täyttyy uudelleen pullossa olevalla juomalla. Tämä ratkaisu vaatii kuitenkin pullon olevan ylösalaisin jonkinlaisessa telineessä tai esimerkiksi juuri katossa roikkumassa. Tämän lisäksi on olemassa kaatonokkia, jotka sulkeutuvat automaattisesti kun juomaa on virrannut tarpeeksi. Niissä voi esimerkiksi olla sisällä pallo, joka nosteen ansiosta nousee tietyllä nopeudella ylöspäin ja lopulta tukkii juoman virtauksen \cite{Barproducts}. Tällaisia kaatonokkia on ollut testattavana Pullonkaula ry:llä, mutta niitä ei olla koskaan saatu käytettyä luotettavasti, ja näin niiden käytöstä on luovuttu.

Nesteiden kaatamista on tutkittu paljon metallin valannan yhteydessä. Siinä on tarpeellista kaataa nestemäisessä olomuodossa olevaa metallia sisältävästä astiasta materiaalia muottiin tarkasti astiaa kallistamalla. Kaadoilta vaaditaan nopeutta ja tarkkuutta sekä virtaavan metallin määrän että muottiin osumisen suhteen. Robotin käyttö on tällaisessa tapauksessa hyödyllistä, koska sula metalli luo vaarallisen työympäristön. Robotille voisi opettaa jokaisen kaatotapauksen erikseen, mutta se on aikaa vievää ja hankalaa, sekä altista muuttujille. Sen takia on kehitetty edistyneempiä tapoja hallita kaatoa robotilla. Tähän useampaa eri ratkaisua ovat tutkineet mm. Noda ja Terashima.

jotka käyttivät laajennettua Kalman-suodatinta nesteen pinnankorkeuden arvioimiseen kaatoastian suulla ja kaadettavan nesteen painon arvioimiseen. Tässä tapauksessa kaadettavan nesteen paino saatiin punnitsemalla kaatoastiaa.

Yksi tapa saada juomien kaadosta tarkka, on mitata kaadettua määrää jollain tavalla. Tätä määrää voidaan käyttää ikään kuin takaisinkytkentänä, jolloin voidaan lopettaa juoman kaataminen, kun juomaa on kaadettu tarpeeksi. Yaskawan HC-10 on monitoimirobotti, mikä tarkoittaa sitä, että sen akseleissa on voima-anturit. Näitä voima-antureita käytetään esimerkiksi törmäysten havaitsemiseksi, jolloin robotti voi törmättyään pysähtyä, tai siihen että operaattori voi käsin työntämällä ohjata robottia. Yaskawan HC-10:n käsiohjaimen käyttöliittymässä on sivu, jolta näkee kaikkiin akseleihin sekä robotin päähän kohdistuvan voiman eri koordinaattisuunnissa reaaliaikaisesti. Nämä voimat on laskettu kaikkien kuuden voima-anturin datan, sekä akselien asennon pohjalta. \cite[s.10-11]{Yaskawa} Aluksi hypoteesina oli, että näiden voima-anturien dataa voisi käyttää robotin tarttujassa olevan massan mittaamiseen. Tämän mittaamisen voisi integroida osaksi robotin ohjelmaa, jolloin robotti pystyisi itse saamaan tiedon siitä, paljonko juomaa tarttujassa olevasta pullosta on jo kaadettu. Tämän ratkaisun  tuli itse voima-anturien tarkkuus. Esimerkiksi pystysuoraan koordinaattisuuntaan vaikuttava voima heittelee jatkuvasti usean Newtonin verran käyttöliittymästä nähtävän datan perusteella. Lisäksi epätarkkuutta lisää se, jos robottikäsi tai pullossa oleva juoma on vähänkään muuttuvassa liikkeessä johonkin suuntaan.

Toinen tapa mitata kaadettua määrää on yksinkertaisesti asettaa jonkinlainen painoanturi tai vaaka kohteena olevan mukin alle ja käyttää sen antamaa lukemaa apuna. Tässä työssä on valittu tämä tapa ratkaisuksi kaatomäärän tarkkuuden parantamiseen. Sen etuna on sen yksinkertaisuus, koska se ei aseta suuria vaatimuksia käytettävälle robotille tai robottisolulle. Lisäksi tämän ratkaisun käyttöä helpottaa se, että Drinkkirobotin robottisolussa on jo käytössä vaaka pullonvaihtopisteellä, ja se kommunikoi robotin logiikan kanssa. Tätä vaakaa voidaan käyttää testaamaan tavan toimivuutta ja sitten jälkeenpäin hankkia kaikkien kaatopisteiden alle erilliset vaa'at ja käyttää niillä samaa tekniikkaa.
